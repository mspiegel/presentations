\documentclass{beamer}
\usetheme{metropolis}
\usecolortheme{owl}
\title{Who is more foolish?}
\subtitle{Novice traps in Rust}
\author{Michael Spiegel}
\date{September 29, 2016}
\begin{document}

\begin{frame}
\titlepage
\end{frame}

\begin{frame}
When I first started learning Rust I started implementing different search trees.
Here are some of the pitfalls I ran into. Hopefully my experience will help you.

{\color{blue} \url{https://www.youtube.com/watch?v=x0ow4X8tiMI}}
\end{frame}

\begin{frame}[fragile]
\frametitle{In the Beginning: Pain}
\begin{verbatim}
struct Node {
  key: i32,
  val: i32,
  color: Color,
  left: Option<Box<Node>>,
  right: Option<Box<Node>>,
}
\end{verbatim}
\end{frame}

\begin{frame}[t,fragile]
\frametitle{1st Attempt: Type Alias}
The type keyword lets you declare an alias of another type. The alias does not declare a new type.
\begin{verbatim}
type Link = Option<Box<Node>>;

impl fmt::Display for Link {
  ... 
}
\end{verbatim}
\end{frame}

\begin{frame}[t,fragile]
\frametitle{1st Attempt: Type Alias}
The type keyword lets you declare an alias of another type. The alias does not declare a new type.
\begin{verbatim}
type Link = Option<Box<Node>>;

impl fmt::Display for Link {
  ... 
}
\end{verbatim}
{\scriptsize \texttt{{\color{red} error}: the impl does not reference any types defined in this crate; only traits defined in the current crate can be implemented for arbitrary types [E0117]}}
\end{frame}

\begin{frame}[t,fragile]
\frametitle{2nd Attempt: Newtype Pattern}
The tuple struct is a hybrid between a tuple and a struct. A 1-dimensional tuple struct is called the `newtype' pattern.
\begin{verbatim}
 struct Link(Option<Box<Node>>);
\end{verbatim}
\end{frame}

\begin{frame}[t,fragile]
\frametitle{2nd Attempt: Newtype Pattern}
The tuple struct is a hybrid between a tuple and a struct. A 1-dimensional tuple struct is called the `newtype' pattern.
\begin{verbatim}
 struct Link(Option<Box<Node>>);
\end{verbatim}
The ownership semantics are 2x as complicated. Am I borrowing the tuple or the value inside the tuple? Reddit thread just on this topic {\color{blue} \url{https://www.reddit.com/r/rust/comments/2cmjfn/how_to_do_typedefsnewtypes}}
\end{frame}

\begin{frame}[t,fragile]
\frametitle{3rd Attempt: Wrapper struct}
The dumbest solution is to wrap the type inside a struct.
\begin{verbatim}
struct Link {
  link: Option<Box<Node>>,
}
\end{verbatim}
\end{frame}

\begin{frame}[t,fragile]
\frametitle{3rd Attempt: Wrapper struct}
The dumbest solution is to wrap the type inside a struct.
\begin{verbatim}
struct Link {
  link: Option<Box<Node>>,
}
\end{verbatim}
This works but the code is cluttered with ``.link" everywhere. The code is harder to understand.
\end{frame}

\begin{frame}
\frametitle{Reboot}
So what is the idiomatic way to express this type?
\end{frame}

\begin{frame}[fragile]
\frametitle{Vanilla Rust}
\begin{verbatim}
struct Node {
  key: i32,
  val: i32,
  color: Color,
  left: Option<Box<Node>>,
  right: Option<Box<Node>>,
}
\end{verbatim}
\end{frame}

\begin{frame}[fragile]
\frametitle{Use Traits}
\begin{verbatim}
trait OptionBoxNode {
  fn get(&self, key: i32) -> Option<i32>;
  fn is_red(&self) -> bool;
  fn color(&self) -> Color;
}

impl OptionBoxNode for Option<Box<Node>> {
  ...
}
\end{verbatim}
\end{frame}

\begin{frame}[fragile]
\frametitle{Escape Hatch}
Newtype when necessary.
\begin{verbatim}
struct Link<'a>(&'a Option<Box<Node>>)

impl<'a> fmt::Display for Link<'a> {
  ...
}
\end{verbatim}
\end{frame}

\begin{frame}
\frametitle{Pro Tips}
The following idioms are helpful...
\end{frame}

\begin{frame}[fragile]
\frametitle{Multiple impl sections}
Declare impls for both the definite type and the optional type. Write functions in the appropriate section depending on whether it manipulates a node or a ``nullable" node.
\begin{verbatim}
impl Node {
  ...
}

impl OptionBoxNode for Option<Box<Node>> {
  ...
}
\end{verbatim}
\end{frame}

\begin{frame}[fragile]
\frametitle{Helper functions everywhere}
\begin{verbatim}
impl OptionBoxNode for Option<Box<Node>> {

fn reference(&self) -> &Box<Node>
  {  self.as_ref().unwrap()  }

fn mutate(&mut self) -> &mut Box<Node>
  {  self.as_mut().unwrap() }

fn left(&self) -> &Option<Box<Node>>
  { &self.as_ref().unwrap().left }
  
}
\end{verbatim}
\end{frame}

\begin{frame}[fragile]
\frametitle{Beware the let ref}
4 ways to declare a variable.
\begin{columns}
\begin{column}{0.5\textwidth}
\begin{itemize}
\item \verb#let x : i32#
\item \verb#let mut x : i32#
\item \verb#let ref x : i32#
\item \verb#let mut ref x : i32#
\end{itemize}
\end{column}
\begin{column}{0.5\textwidth}
\end{column}
\end{columns}
\end{frame}

\begin{frame}[fragile]
\frametitle{Beware the let ref}
2 sane ways to declare a variable.
\begin{columns}
\begin{column}{0.5\textwidth}
\begin{itemize}
\item \verb#let x : i32;#
\item \verb#let mut x : i32;#
\item {\color{red} \verb#let ref x : i32;#}
\item {\color{red} \verb#let mut ref x : i32;#}
\end{itemize}
\end{column}
\begin{column}{0.5\textwidth}
\begin{itemize}
\item \verb#let x : i32;#
\item \verb#let mut x : i32;#
\item \verb#let x : &i32;#
\item \verb#let mut x : &i32;#
\end{itemize}
\end{column}
\end{columns}
\end{frame}

\begin{frame}
\frametitle{Bonus content: RustConf 2016}
RustConf was on September 10 in Portland OR.
\begin{itemize}
\item Slides and exercises from the tutorials \url{http://rust-tutorials.com/RustConf16}
\item Opening keynote. ``Fast, reliable, productive" -- Pick three
\item Josh Triplett on the RFC process
\item If you use Rust commercially, community-team@rust-lang.org wants your feedback
\end{itemize}
Rustbelt Rust coming up October 27-28 in Pittsburg PA.
\end{frame}

\begin{frame}
\frametitle{Putting everything together}
\begin{itemize}
\item \url{https://github.com/mspiegel/rust-yaar}
\item \url{https://github.com/rust-lang-nursery/rustup.rs}
\item \url{https://github.com/rust-lang-nursery/rustfmt}
\item \url{https://github.com/Manishearth/rust-clippy}
\end{itemize}
\end{frame}

\end{document}